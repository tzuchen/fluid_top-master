\documentclass{article}

\usepackage{graphics}
\usepackage{amsmath}
\usepackage{amsthm}
\usepackage{amsfonts}

\title{The Plan of Four Papers}
\author{Tzu-Chen Liang}
\date{\today}


\begin{document}

\maketitle

\begin{abstract}
This is the plan of the four papers I am going to write.

\end{abstract}

%%%%%%%%%%%%%%%%%%%%%%%%%%%%%%%%%%%%%%%%%%%%%%%%%%%%%%%%%%%%%%%%%%

\section{Optimized Mixing in Microfluidic Channels}

 \begin{itemize}
   \item TAO!
 \end{itemize}
The goal of this research is to design a passive microfluidic mixer via topology optimization. It is very difficult to mix solutions in a pressure-driven microfluidic channel. The flow tends to be laminar and hence the molecular diffusion happens inefficiently. We optimize the shape of the structure inside the channel to accelerate the mixing. Structures that can stir the flow to produce transverse velocity components and enhance the mixing in a passive way are proposed by Abrahm D. Strook et al. In this research, we present a stochastic model to simulate the mixing process happening in a periodical structured channel when the velocity field is fully developed and also periodical. We then show the complicated mixing phenomenon is similar to a relatively simple Markov process and their mixing rates are related. We further optimize the shape of the structure by a sub-gradient method and accelerate the mixing significantly.

%%%%%%%%%%%%%%%%%%%%%%%%%%%%%%%%%%%%%%%%%%%%%%%%%%%%%%%%%%%%%%%%%%
\section{Mixing Channel Example}

 \begin{itemize}
   \item Application section
   \item Example 1 Mixing channel
   \item Example 2 Standard map
   \item Discussion
 \end{itemize}

This research is to answer a more fundamental optimal predictor question: what is the best Markov chain model for a deterministic/stochastic system with deterministic/stochastic finite observations. This model reduces to a first-order accuracy linear model when both the system and the observations are deterministic. However, it becomes much more useful when one or both of them are stochastic. We thus use the result to build a finite dimensional model for the mixing channel.   

%%%%%%%%%%%%%%%%%%%%%%%%%%%%%%%%%%%%%%%%%%%%%%%%%%%%%%%%%%%%%%%%%%
\section{The Cutoff Phenomenon of Passive Scalars Advected by Chaotic Maps with Small Diffusion}

 \begin{itemize}
   \item Numerical examples.
   
 \end{itemize}

We numerically study the decay of the variance of a passive scalar function in a chaotic flow when the diffusibility tends to zero and link the result to the well-known cutoff phenomenon found in Markov chain simulations. An efficient and parallelizable Markov chain model is built to simulate the chaotic map with small diffusion. The sequence of Markov chains generated by reducing the numerical diffusion or improving the resolution presents a cutoff when the underlying map is chaotic. This result points out one possibility of the source of cutoff phenomenon and also serves as a theoretical fundation of the mixing channel design problem.



%%%%%%%%%%%%%%%%%%%%%%%%%%%%%%%%%%%%%%%%%%%%%%%%%%%%%%%%%%%%%%%%%%
\section{The Properties of Chaotic Map in Mixing}

 \begin{itemize}
   \item the symmetric property!
 \end{itemize}

The mixing properties of chaotic maps are not well understood so far. In this research we try to suggest an alternative possiblility which causes the multi-stage feature of the mixing variance trajectory.   


\end{document}
