%
% Abstract
%
This thesis studies chaotic mixing induced by microfluidic channels and builds the relation between it and the cutoff phenomenon in finite Markov Chains. We develop a topology optimization methodology for optimizing the shape of pressure-driven microfluidic channels to maximize the passive mixing rate of advected tracers. The optimization procedure uses a relaxation of Stokes flow by allowing a permeable structure, and the objective can be a function of either the fluid velocity field or the particle map from inlet to outlet. We present two new channel
designs, one that is an optimized version of the herringbone mixer of Stroock et al., and one that mixes more quickly by using a fully 3-D structure in the channel center. These channels deliver approximately 30\% and 60\% reductions in the 90\% mixing lengths. To compare our numerical simulations to experiments we approximate the inlet-outlet particle map by a Markov Chain and show that this cheaply approximates the true stochastic map.

We then numerically study the decay of the variance of a passive scalar function advected by the Standard Map when the diffusion goes to zero. The Markov Chain model we developed for microfluidic mixing channel simulation is applied here with very high resolution (up to $6.4 \times 10^9$ states) to approximate near-zero diffusion. Our numerical evidence shows that the mixing trajectories of the Standard Map can be characterized by using the cutoff phenomenon for finite Markov Chains.

In the last part of this thesis, we apply the definition of the cutoff phenomenon to the study of the evolution of a probability density function by 1-D chaotic maps. A new object called a stochastic symbol sequence is developed to prove that for a set of initial distributions, the total variation versus iteration curves present cutoffs. Moreover, we can generate a set of initial probability distributions such that when evolved by chaotic maps, they present the same limit behavior as the cutoff sequences found in specific finite Markov Chains. The results can be applied to any 1-D chaotic map that has full symbolic dynamics.
