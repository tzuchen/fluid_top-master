%
% Introduction of Chapter topopt 
%


We develop a topology optimization methodology for optimizing the
shape pressure-driven microfluidic channels to maximize the passive
mixing rate of advected tracers. The optimization procedure uses a
relaxation of Stokes flow by allowing a permeable structure and the
objective can be a function of either the fluid velocity field or the
particle map from inlet to outlet. We present two new channel designs:
one that is an optimized version of the herringbone mixer of Stroock
et al., and one that mixes more quickly by using a fully 3D structure
in the channel center. These channels deliver approximately 30\% and
60\% reductions in the 90\% mixing lengths. To compare our numerical
simulations to experiments we approximate the inlet-outlet particle
map by a Markov Chain and show that this cheaply approximates the true
stochastic map.

%In this chapter, we focus on the design and optimization of passive microfluidic mixing channels. A microfluidic mixing channel typically has
%cross-section dimension $ell\sim 100\mu m$, and Reynolds number $\text{Re}=U\ell/\nu$ is less than $100$ \cite{Stroock2002} ($U$ is the average
%velocity of the liquid and $\nu$ is the kinematic viscosity of the fluid).  Fluid flow on this scale is highly laminar and the mixing of
%materials between streams is purely diffusive. The dimensionless number that controls the length of the channel required for mixing is
%P\'{e}clet number ($\text{Pe}= U\ell/D$ where $D$ is the molecular diffusivity). For a pressure-driven mixing channel the mixing length can be
%expected to grow linearly with $\text{Pe}$ and is usually much more than $1\,\text{cm}$. Hence various designs are proposed to stir the
%flow inside the channel and produce transverse velocities to enhance the mixing \cite{Stroock2002, Ottino2004Science, Wiggins2004}.

%The mixing problem has been linked to chaotic mixing protocols, which
%are believed to have the best mixing results due to the stretching and
%folding features of chaotic maps. A typical way to realize chaotic
%mixing is through the design of linked twist maps, and has been
%studied in \cite{Wiggins2004}. However, there is no direct way to
%realize the designed linked twist map in a mixing channel, either by
%passive structure or active mechanisms such as variable-frequency
%pumps or internal moving components. In this paper, we use the
%techniques developed in topology optimization to find the internal
%structure of a mixing channel to realize a desired flow field or flow
%map. The results can be applied to mixing channel design when the
%desired mixing protocol is known.


%Topology optimization has been applied to the design of optimal shapes
%of pipes or diffusers such that the total potential power drop is
%minimized \cite{Evgrafov2005, Borrvall2003}. Darcey flow is used to
%simulate the flow inside porous material and form a relaxation. The
%design parameters in this formulation are the permeability of the
%material on a spatial grid. In this case, the relaxed problem is still
%non-convex, but can be solved by means of sequential separable and
%convex programming. A two-step solution procedure is thus applied to
%find a black/white solution with permeability either $0$ or infinity
%at a point .

%In this chapter we use the same relaxation strategy to formulate the
%mixing channel design as a topology optimization problem. The
%objective function we use is a function of the velocity field or a map
%between the inlet and the outlet of one period of the channel. It is a
%nonlinear optimization problem and a sub-gradient method is developed
%to find the local minimum of the objective function. We do not
%consider the fabrication issue explicitly when solving the topology
%optimization problem.

In section \ref{sec:opt} we describe the mathematical model of mixing
channels and how to form the topology optimization problem using a
relaxation of Stokes flow. Section \ref{sec:simu} discusses the
simulation issue, and a Markov chain model is proposed to approximate
the solution of the advection-diffusion equation to evolve the color
intensity field. Results are given in section \ref{sec:topoptresults} and 
conclusions are in section \ref{sec:topoptconclusion}.
