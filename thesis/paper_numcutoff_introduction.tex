%
% Introduction
%
\section{Introduction}
\label{sec:numcutoffintro}

The question of how chaotic advection mixes a passive scalar function
has attracted much research effort in recent years
\cite{Ottino2004}. The main issues in this field are: how to measure
the thoroughness of the mixing, how the mixing process changes
qualitatively and quantitatively when the diffusion is close to zero,
and how to enhance the overall mixing process by designing the map
which produces chaotic advection. Unfortunately, we have only partial
understanding for most of these topics. In spite of the fact that the
detailed mechanism of mixing is unclear, non-trivial mixing processes
have been observed in experiments \cite{Rothstein1999, Voth2002} and
can be simulated by large-scale computations \cite{topopt,
  Tsang2005}. In this article we build a linear Markov Chain model to
simulate chaotic mixing with small diffusion. Similar approaches for
nonlinear dynamical systems can be found in, for example,
\cite{Dellnitz1999, Dellnitz2002, Froyland1998, Froyland1999,
  Froyland2001} . This simple and parallelizable linear model not only
captures the multi-stage feature of a chaotic mixing process, but also
generates a series of finite Markov Chains through which we can
observe the multi-stage feature of the mixing trajectory near the
zero-diffusion limit.

\paragraph{The multi-stage feature of chaotic mixing processes.}
A widely observed phenomenon in the chaotic mixing process when small
diffusion exists is the two or three-stage transition
\cite{Thiffeault2003-13, Fereday2002, Antonsen1996, Mezic2005}. The
map does not mix the scalar function with a constant rate in
general. When the variance of the scalar function is measured during
the mixing process, one can in general observe a relatively flat decay
initially, followed by a super-exponential change, and then finally it
tends to a exponential decay. We are interested in when these
transitions happen, why they happen, and how to predict the slope of
the exponential region. A good review and physical interpretation can
be found in \cite{Thiffeault2004}.

Thiffeault and Childress \cite{Thiffeault2003-13} study these
properties for a modified Arnold's cat map. Analytical formulas are
given to predict the transitions as well as the slopes. Because the
linear part of this map has an eigenvalue 2.618, which stretches very fast, 
and the chaotic part is relatively small, the three phases are
separated clearly. The same analytical procedure cannot be applied to,
for example, the Standard Map, although the only difference between the
Standard Map and the modified Arnold's cat map is in the linear part.

\todo{MW: Check rewording of sentence in above paragraph starting
  ``Because the linear part \ldots''}

As for the exponential decay part, there is still debate about whether
the decay rate goes to zeros in the zero diffusivity limit or whether
it tends to a constant independent of the diffusion
\cite{Thiffeault2004, Tsang2005}. Theoretical analysis shows both of
these possibilities can occur for different chaotic flows
\cite{Haynes2005}.

Difficulties typically arise in studying the above problems
numerically, because the small diffusion usually means fine grids are
required in the solution of the advection-diffusion equation or the
simulation of the map. Some studies and numerical results conclude that a
proportional relation exists between the stationary decay rate and the
diffusion \cite{Cerbelli2003, Pikovsky2003}. However, this is only true for
certain diffusion ranges.  



In \cite{Tsang2005}, the author
uses a simple and parallelizable numerical strategy which can simulate
up to $6 \times 10^4$ by $6 \times 10^4$ grids to show the decay rate
of a certain chaotic map tends to a constant. However, general
numerical results for most other chaotic maps have not been found so
far.

\todo{MW: I reworded the sentence starting ``Some simulation results
  \ldots''. Is this true? Did some people really incorrectly conclude
  that a proportional relationship exists between the stationary decay
  rate and the diffusion? And did they incorrectly conclude this only
  because they didn't use enough resolution? Are \cite{Cerbelli2003,
    Pikovsky2003} the people who made such an error or are they the
  people who pointed out the error?}
  
\todo{TC: In both \cite{Cerbelli2003, Pikovsky2003}, there are analytical results to compare with their numerical simulations. So I believe their conclusions are correct--the proportional relation between diffusion and stationary decay rate. However, they deal with the cases that the diffusions are relatively large ($D=10^{-2} \sim 10^{-4}$ in \cite{Cerbelli2003}, and $500 \times 500$ grids ($D\approx 10^{-4}$ when adding diffusion) in \cite{Pikovsky2003}). 

In \cite{Cerbelli2003} the author mentions that when $D$ goes to zero, decay rate should be a constant. In \cite{Pikovsky2003}, they simply say the resolution is not enough so they cannot conclude the zero-diffusion decay rate. 

Hence, to be safe I reword the above paragraph again. 
} 

  

In this paper we focus on the mixing process of the Standard Map in
the near-zero diffusion limit. We present a numerical strategy to
simulate the map with very high resolution (up to $8 \times 10^4$ by
$8 \times 10^4$ grids) and hence very low numerical diffusion. This
numerical strategy is realized by a Markov Chain simulation. To
characterize the evolution of the scalar variance in the near-zero
diffusion limit, we use the concept of cutoff from the study of finite
Markov Chains. We present numerical evidence to suggest that the
sequence of models presents a cutoff, which qualitatively
characterizes the mixing process when the diffusion goes to zero. The
main contribution of this paper is to build a bridge between finite
Markov Chain theory and $2$-D chaotic maps with small
diffusion. Related analytical results for $1$-D chaotic maps
clarifying their relationship to the cutoff phenomenon can be found in
\cite{symdyn}.

This paper is organized as the follows: in section
\ref{sec:numcutoffbackground} we briefly review the background of the
operators we use and the cutoff phenomenon. Section
\ref{sec:modelreduction} provides a model reduction view of building
the Markov Chain model of chaotic maps. Section \ref{sec:numstrategy}
talks about the numerical strategies we use in simulations. Numerical
evidence of Standard Map cutoff is given in section
\ref{sec:numresults} and conclusions are presented in section
\ref{sec:numcutoffconclusion}.
