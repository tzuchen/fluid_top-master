%
% Introduction
%
\section{Introduction}
\label{sec:symdynintro}

How many riffle shuffles are required to sufficiently mix 52 cards? This question have been
answered by Bayer and Diaconis in \cite{Diaconis1992}. It is quite surprising that the cards are
highly ordered in the first several shuffles (6 for 52 cards) and then randomized almost abruptly.
This kind of sharp change in some measure of order/disorder has been discovered in many finite
Markov Chains, and named cutoff phenomenon \cite{Diaconis1986}. A recent review about the cutoff
phenomenon of random walks on finite groups can be found in \cite{LSaloff-Costt2004}. On the other
hand, chaotic mixing process is observed to have similar
properties \cite{Thiffeault2003-13, Thiffeault2004, Tsang2005}. Think of the cream poured
into coffee, while stirring, the cream keeps stretching for a while and the mixing with coffee
happens suddenly. Chaotic mixing has been applied in, for example, microfluidic mixing channel
design \cite{Ottino2004Science, Wiggins2004, Ottino2004}, random search strategies and
generating random numbers. However, it is still unclear how to characterize and measure the mixing
process. In this article, we try to build the relation between cutoff phenomenon and the chaotic
mixing process by generalizing symbolic dynamics of chaotic maps. The result shows ``cutoff'' in
the mixing process can be produced by chaotic maps with full symbolic dynamics and certain initial
conditions.

In the next section, we briefly review cutoff phenomenon and point out how a simple chaotic map can produce ``cutoff''. Section \ref{sec:symdyn} we
introduce symbolic dynamics. We define a new object called stochastic symbol sequence, which serves as the main tool to build the bridge
between cutoff phenomenon and chaotic mixing. The main result is given in section \ref{sec:mainresults}, and finally a conclusion in section \ref{sec:symdynconclusion}.


