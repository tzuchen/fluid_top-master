%
% Introduction
%
\section{Introduction}
\label{sec:topoptintro}

\todo{MW: when citing in a paragraph \cite{DavidJ2002, Stroock2002}
  like this, we leave a space before the cite and put multiple
  citations in the same cite separated by commas.}

\todo{MW: inside paragraph titles we use mixed case (first word
  capitalized, then all lowercase).}

\todo{MW: Following Knuth, we should do units like this: $p =
  20\,\text{kg}\,\text{m}\,\text{s}^{-1}$ or $v =
  10\,\text{m}/\text{s}$. I tend to prefer using the first form.}

\paragraph{Microfluidic mixing channels.}
Microfluidic systems control and manipulate liquids in microliter or nanoliter amounts. The study of microfluidics emerged in the 1990s and
is now widely applied in various fields such as the development of DNA chips \cite{Burns1998}, molecular biology \cite{DavidJ2002},
chemical reactions \cite{Andersson2000}, transfers of small volumes of materials \cite{Sammarco1999}, and lab-on-a-chip technology
\cite{weigl2003,Stone2004}. One of the challenges in microfluidics is the design of mixing channels, whose objective is to thoroughly mix
two or more different liquids. Although the mixers designed by using active components like micro-pumps to stir the flow have very
promising results \cite{Yang2000,Deshmukh2000}, passive mixing devices have advantages in manufacturing simplicity and price.

In this paper, we focus on the design and optimization of passive microfluidic mixing channels. A microfluidic mixing channel typically has
cross-section dimension $\ell\sim 100\mu m$, and Reynolds number $\text{Re}=U\ell/\nu$ is less than $100$ \cite{Stroock2002} ($U$ is the average
velocity of the liquid and $\nu$ is the kinematic viscosity of the fluid).  Fluid flow on this scale is highly laminar and the mixing of
materials between streams is purely diffusive. The dimensionless number that controls the length of the channel required for mixing is
P\'{e}clet number ($\text{Pe}= U\ell/D$ where $D$ is the molecular diffusivity). For a pressure-driven mixing channel the mixing length can be
expected to grow linearly with $\text{Pe}$ and is usually much more than $1\,\text{cm}$. Hence various designs are proposed to stir the
flow inside the channel and produce transverse velocities to enhance the mixing \cite{Stroock2002, Ottino2004Science, Wiggins2004}.

The mixing problem has been linked to chaotic mixing protocols, which
are believed to have the best mixing results due to the stretching and
folding features of chaotic maps. A typical way to realize chaotic
mixing is through the design of linked twist maps, and has been
studied in \cite{Wiggins2004}. However, there is no direct way to
realize the designed linked twist map in a mixing channel, either by
passive structure or active mechanisms such as variable-frequency
pumps or internal moving components. In this paper, we use the
techniques developed in topology optimization to find the internal
structure of a mixing channel to realize a desired flow field or flow
map. The results can be applied to mixing channel design when the
desired mixing protocol is known.


\paragraph{Topology optimization.}
A typical topology optimization problem is to distribute a given amount of material in a design domain subject to load and support
conditions such that the stiffness of the structure is maximized \cite{Bendsoe2003}. The design parameters are usually the spatial
distribution of the material. Even though the optimal solution is in general quite sparse in space, the number of variables is often
inevitably large in the formulation. Some problems, such as truss topology design, can be formulated as a convex optimization
problems \cite{BenTal1997} by relaxing the material density to be progressive and solved by efficient algorithms. The optimal solution of
the relaxed optimization is shown to be a black/white solution and thus also the optimal solution of the original problem. In other cases
when the problem has no convex formulation, nonlinear optimization techniques need to be applied and it is in general hard.

\todo{MW: We should phrase the use of Darcey flow models as a
  relaxation of the true optimization problem. Do the papers below use
  $L^1$ norms or something to ensure that the optimum solution is
  black/white? We should say something clearer about when the optimum
  of the relaxed problem is also the optimum of the unrelaxed
  problem.}

\todo{TC: The relaxation and the non-convexity are two different but
related issues. In truss design, we have convex relaxation, and so the
solution is optimal for both relaxed and original problem. In the
fluid problem, even we do relaxation, it is sill non-convex.}

\todo{TC:``by means of sequential separable and convex programming''
is the terminology used by the paper. It means at each iteraiton, the
permeability is constrained in some small range and the objective
function is expanded locally as a linear function to make the problem
convex}

\todo{TC: ``the two-step solution procedure'' also used in the paper:
first step, find a gray solution, second step, use the solution as a
initial guess and tune a parameter to make the optimization to tend to
form a black/white solution. They claim this is to avoid local
optimum.}

Topology optimization has been applied to the design of optimal shapes
of pipes or diffusers such that the total potential power drop is
minimized \cite{Evgrafov2005, Borrvall2003}. Darcey flow is used to
simulate the flow inside porous material and form a relaxation. The
design parameters in this formulation are the permeability of the
material on a spatial grid. In this case, the relaxed problem is still
non-convex, but can be solved by means of sequential separable and
convex programming. A two-step solution procedure is thus applied to
find a black/white solution with permeability either $0$ or infinity
at a point .

In this paper we use the same relaxation strategy to formulate the
mixing channel design as a topology optimization problem. The
objective function we use is a function of the velocity field or a map
between the inlet and the outlet of one period of the channel. It is a
nonlinear optimization problem and a sub-gradient method is developed
to find the local minimum of the objective function. We do not
consider the fabrication issue explicitly when solving the topology
optimization problem.

\todo{MW: I'm a bit nervous about the following sentence, so maybe we
  should just leave it out: ``However, the resultant structures are
  reasonable to fabricate by current technology.''}

\todo{TC: Agree.}

\paragraph{The simulation of Advection-Diffusion equation.}
Another challenge in the design of microfluidic mixing channels is
that there is no clear measure of how well a channel mixes. In
experiments it is common to measure the variance of colored liquids on
the cross-section of the channel to see how well they are mixed
\cite{Stroock2002}. However, when doing simulation, this corresponds
to solve the Advection-Diffusion equation for a 3-D flow field, which
is expensive. The problem that there is no clear link between the
variance of the liquid on a cross-section and the channel structures
remains hard so far. In this paper, we develop a Markov Chain model to
approximate the mixing process.  Similar approaches can be found in,
for example, \cite{Dellnitz1999, Dellnitz2002, Froyland1998,
Froyland1999, Froyland2001}. It is a cheap way to replace the solving
of Advection-Diffusion equation and can let one observe the mixing
process and measure the variance of the colored field easily. This
model is by no means an accurate solution of the Advection-Diffusion
equation, but it captures the most important factors of the chaotic
mixing: stretching, folding and molecular diffusion.

\paragraph{Outline.}
In section \ref{sec:opt} we describe the mathematical model of mixing
channels and how to form the topology optimization problem using a
relaxation of Stokes flow. Section \ref{sec:simu} discusses the
simulation issue and a Markov chain model is proposed to approximate
the solution of the Advection-Diffusion equation to evolve the color
intensity field. Results are given in section \ref{sec:topoptresults} and
finally conclusions are in section \ref{sec:topoptconclusion}.
