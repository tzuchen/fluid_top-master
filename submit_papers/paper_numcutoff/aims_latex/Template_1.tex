% !!!IMPORTANT NOTE: Please read carefully all information including those preceded by % sign
\documentclass{aims}
\usepackage{amsmath}
  \usepackage{paralist}
  \usepackage{graphics} %% add this and next lines if pictures should be in esp format
  \usepackage{epsfig} %For pictures: screened artwork should be set up with an 85 or 100 line screen
 \usepackage[colorlinks=true]{hyperref}
   % Warning: when you first run your tex file, some errors might occur, please just
   % press enter key to end the compilation process,  then it will be fine if you run your tex file again.
   % Note that it is highly recommended by AIMS to use this package.
\hypersetup{urlcolor=blue, citecolor=red}
%\usepackage{hyperref}

  \textheight=8.2 true in
   \textwidth=5.0 true in
    \topmargin 30pt
     \setcounter{page}{1}

% The next 5 line will be entered by an editorial staff.
\def\currentvolume{X}
 \def\currentissue{X}
  \def\currentyear{200X}
   \def\currentmonth{XX}
    \def\ppages{X--XX}

 % Please minimize the usage of "newtheorem", "newcommand", and use
 % equation numbers only situation when they provide essential convenience
 % Try to avoid defining your own macros

\newtheorem{theorem}{Theorem}[section]
\newtheorem{corollary}{Corollary}
\newtheorem*{main}{Main Theorem}
\newtheorem{lemma}[theorem]{Lemma}
\newtheorem{proposition}{Proposition}
\newtheorem{conjecture}{Conjecture}
\newtheorem*{problem}{Problem}
\theoremstyle{definition}
\newtheorem{definition}[theorem]{Definition}
\newtheorem{remark}{Remark}
\newtheorem*{notation}{Notation}
\newcommand{\ep}{\varepsilon}
\newcommand{\eps}[1]{{#1}_{\varepsilon}}


%% Place the running title of the paper with 40 letters or less in []
 %% and the full title of the paper in { }.
\title[Running heading with forty characters or less]
      {The Full Title of Your Paper}

% Place all authors' names in [ ] shown as running head;
% No more than 40 letters. Leave { } empty
% Please use `and' to connect the last two names if applicable
\author[first-name1 last-name1 and first-name2 last-name2]{}

% It is required to enter MSC and Keywords.
\subjclass{Primary: 58F15, 58F17; Secondary: 53C35.}
 \keywords{Dimension theory, Poincar\'e recurrences, multifractal analysis.}

% Email address of each of all authors is required.
% You may list email addresses of all other authors, separately.
 \email{email1@smsu.edu}
 \email{email2@aimSciences.org}
 \email{email3@ece.pdx.edu}

% Put your short thanks below. For long thanks/acknowlegements,
%please go to the last acknowlegments section.
\thanks{The first author is supported by NSF grant xx-xxxx}

\begin{document}
\maketitle

% Enter the first author's name and address:
\centerline{\scshape First-name1 last-name1 }
\medskip
{\footnotesize
% please put the address of the first author
 \centerline{First line of the address of the first author}
   \centerline{Other lines}
   \centerline{ Springfield, MO 65801-2604, USA}
} % Do not forget to end the {\footnotesize by the sign }

\medskip

\centerline{\scshape First-name2 last-name2 and First-name3
last-name3}
\medskip
{\footnotesize
 % please put the address of the second  and third author
 \centerline{ First line of the address of the second author}
   \centerline{Other lines}
   \centerline{Springfield, MO 65810, USA}
}

\bigskip

% The name of the associate editor will be entered by an editorial staff
% "Communicated by the associate editor name" is not needed for special issue.
 \centerline{(Communicated by the associate editor name)}


%The abstract of your paper
\begin{abstract}
This is the abstract of your paper and it should not exceed
\textbf{200} words.
\end{abstract}

%The title of your section 1
\section{Introduction}

Please use this AIMS template to prepare your tex file after the
paper is accepted by an AIMS journal. Please read carefully all
information including those proceeded by \% sign. These are
important instructions and explanations. Thank you for your
cooperation.

%The title of your section 2
\section{Examples}

%The title of your first subsection in section 2
\subsection{A sample Theorem}

\begin{theorem} \label{result1}
        Content of your theorem.
\end{theorem}

\begin{proof}
To refer to equations in your paper, use the commands:
\ref{Quotient}, \ref{Multi} and \ref{Equ3}.
\end{proof}

%The title of your second subsection in section 3
\subsection{A sample Lemma}

\begin{lemma} \label{L: Lyapunov exponents} State your lemma here.
\end{lemma}

\begin{proof}
Your proof statements.
\end{proof}

Text in both definition and remark should not be slanted.
%The title of your third subsection in section 2
\subsection{A sample Remark}

\begin{remark}
Content of your remarks.
\end{remark}
%The title of your fourth subsection in section 2
\subsection{A sample Definition}

\begin{definition} Sample: Let $\phi_{t}$ be an Anosmia flow on a
        compact space $V$ and $A \subset V$ a dense set. Say
        that the upper Lacunae exponents are
        \emph{$\frac{1}{2}$-pinched} on $A$ if
\bigskip

  \begin{equation}\label{Quotient}
        \sup_{x \in A} \frac{\max \{ |\bar{\lambda}|: \bar{\lambda} \
        \text{is a nonzero upper Lyapunov exponent at} \ x \}}
        {\min \{ |\bar{\lambda}|: \bar{\lambda} \ \text{is a
        nonzero upper Lyapunov exponent at} \ x\}}
         \leq 2.
  \end{equation}
\end{definition}
\bigskip
%The title of your fifth subsection in section 2
\subsection{Example of inserting a Figure}

\begin{figure}[htp]
\begin{center}
  % Requires \usepackage{graphicx}
  % replace aims_logo.eps by your figure file name
  \includegraphics[width=2in]{aims_logo.eps}\\
  \caption{Here is the Caption of your figure}\label{AIMS}
  \end{center}
\end{figure}


%The title of your section 3
\section{How to align the math formulas}

\begin{theorem} \label{result2}
        Content of your theorem.
\end{theorem}

In the proof below, we would like to show you how to align the math
formulas:
\begin{proof}[Proof of Theorem \ref{result2}]
    Please refer to the following example and align your math formulas:
\begin{equation}\label{Multi}
 \begin{split}
    \eps{\theta} \wedge d\eps{\theta}^{n-1} =& (\theta_0 + \ep \alpha)
    \wedge (d(\theta_0 + \ep \alpha))^{n-1} \quad \text{since } d\alpha = 0\\
    =& (\theta_0 + \ep \alpha) \wedge
    (d\theta_0)^{n-1} + \theta_0 \wedge d\theta_0^{n-1} - \ep d(\alpha \wedge
    \theta_0 \wedge d\theta_0^{n-2})\\
    &  + \theta_0 \wedge d\theta_0^{n-1} + \ep \alpha \wedge
    d\theta_0^{n-1}  \\
    =& \theta_0 \wedge d\theta_0^{n-1} - \ep d(\alpha \wedge
    \theta_0 \wedge d\theta_0^{n-2}),
 \end{split}
\end{equation}

It also can be aligned in the following way:

\begin{equation}\label{Multi}
 \begin{split}
    &\eps{\theta} \wedge d\eps{\theta}^{n-1} \\
    =& (\theta_0 + \ep \alpha)
    \wedge (d(\theta_0 + \ep \alpha))^{n-1} \quad \text{since } d\alpha = 0\\
    =& (\theta_0 + \ep \alpha) \wedge
    (d\theta_0)^{n-1} + \theta_0 \wedge d\theta_0^{n-1} - \ep d(\alpha \wedge
    \theta_0 \wedge d\theta_0^{n-2})\\
    &  + \theta_0 \wedge d\theta_0^{n-1} + \ep \alpha \wedge
    d\theta_0^{n-1}  \\
    =& \theta_0 \wedge d\theta_0^{n-1} - \ep d(\alpha \wedge
    \theta_0 \wedge d\theta_0^{n-2}),
 \end{split}
\end{equation}

\newpage

Here is another example if the math expression in [ ] exceeds one
line:

\begin{equation}\label{Equ3}
 \begin{split}
\int_0^T |u_0(t)|^2dt  \leq& \delta^{-1} [\int_0^T
(\beta(t)+\gamma(t)) dt\\
&  \quad\quad+T^{\frac{2(p-1)}{p}}(\int_0^T
|\dot{u}_0(t)|^pdt)^{\frac{2}{p}}
 +T^{\frac{2(p-1)}{p}}(\int_0^T |\dot{u}_0(t)|^pdt)^{\frac{2}{p}}].
 \end{split}
\end{equation}

 Please use the displaystyle if your formulas fully
occupy a paragraph, while use textstyle among the text.

For two equations:
\begin{align*}
A &= \theta_0 \wedge d\theta_0^{n-1} - \ep d(\alpha \wedge \theta_0 \wedge d\theta_0^{n-2})\\
B&=\theta_1 \wedge d\theta_1^{n-1} - \ep d(\alpha \wedge \theta_1
\wedge d\theta_1^{n-2})
\end{align*}
Please align your formulas nicely according above examples. Thanks.
\end{proof}

%For acknowledgements section, please don't number the section, please begin it with \section*{Acknowledgements}
\section*{Acknowledgments} We would like to thank you for \textbf{following
the instructions above} very closely in advance. It will definitely
save us lot of time and expedite the process of your paper's
publication.

% You may incorporate your references as follows in your main tex file.
% Using BibTex is not recommended but can be handled.

\begin{thebibliography}{99}


\bibitem{A11}
     \newblock FirstName (or FirstNameInitial.) MiddleInitial. LastName, % first name middle initial. and then last name.  Only the first character in the paper title is capitalized.
     \newblock \emph{Title of the paper},
     \newblock Name of the journal, \textbf{Volume} (Year), StaringPage--EndingPage.

% Example of paper with MR number:
\bibitem{A22} (MR2082924)
     \newblock C. Wolf,
     \newblock \emph{A mathematical model for the propagation of a hantavirus in structured populations},
     \newblock Discrete Continuous Dynam. Systems - B, \textbf{4} (2004), 1065--1089.

% Example of multiple authors:
\bibitem{BFL} (MR1124979)
    \newblock Y. Benoist, P. Foulon and F. Labourie, % Use `and' connect the last two authors
    \newblock \emph{Flots d'Anosov a distributions stable et instable
     differentiables},
    \newblock (French) [Anosov flows with stable and unstable differentiable
     distributions], J. Amer. Math. Soc., \textbf{5} (1992), 33--74.

% Example of a book in the reference:
\bibitem{rB} (MR1301779)
     \newblock J. Smoller,
     \newblock ``Shock Waves and Reaction-Diffusion Equations,"
     \newblock 2$^{nd}$ edition, Springer-Verlag, New York, 1994.
% Example of an article in Academic Press or a book:
\bibitem{Serrin} (MR0402274)
    \newblock J. Serrin,
    \newblock  \emph{Gradient estimates for solutions of nonlinear elliptic
                     and parabolic equations},
    \newblock  in ``Contributions to Nonlinear Functional Analysis" (eds. E.H. Zarantonello and Author 2),
                Academic Press, (1971), 33--75.
% Example of a thesis:
\bibitem{Zeng}
    \newblock FirstName LastName,
    \newblock  ``Torsion Cycles and Set Theoretic Complete Intersection,"
    \newblock  Ph.D thesis, Washington University in St. Louis, 2006.

% Example of a preprint article with 7 digits arXiv number:
\bibitem{quas}
\newblock M. Entov, L. Polterovich and F. Zapolsky,
\newblock \emph{Quasi-morphisms and the Poisson bracket},
\newblock  preprint, \arXiv{math/0605406}.

% Example of a preprint article with 8 digits arXiv number:
\bibitem{Te2008}
\newblock A. Teplinsky,
\newblock \emph{Herman's theory revisited},
\newblock preprint, \arXiv{0707.0078}.

% No author names:
\bibitem{Whho}
\newblock
\newblock ``SARS Expert Committee, SARS in Hong Kong: From Experience to
Action," Report of Hong Kong SARS Expert Committee,
\newblock 2003. Available from: \url{http://www.sars-expertcom.gov.hk/english/reports/reports.html}.

% No title:
\bibitem{tachet}
\newblock F. Abergel and R. Tachet,
\newblock
\newblock work in progress.

\end{thebibliography}

\medskip
% The data information below will be filled by AIMS editorial staff
Received xxxx 20xx; revised xxxx 20xx.
\medskip

\end{document}
