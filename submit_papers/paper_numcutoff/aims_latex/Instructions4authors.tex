\documentclass{aims}
\usepackage{amsmath}
  \usepackage{paralist}
  \usepackage{graphicx} %% add this and next lines if pictures should be in esp format
  \usepackage{epsfig} %For pictures: screened artwork should be set up with an 85 or 100 line screen
%\usepackage[pdftex]{graphicx}
%  \usepackage{epstopdf}
 \usepackage[colorlinks=true]{hyperref}
\hypersetup{urlcolor=blue, citecolor=red}

  \textheight=8.2 true in
   \textwidth=5.0 true in
    \topmargin 30pt
     \setcounter{page}{1}

% The next 5 line will be entered by an editorial staff.
\def\currentvolume{X}
 \def\currentissue{X}
  \def\currentyear{200X}
   \def\currentmonth{XX}
    \def\ppages{X--XX}
    \def\DOI{}
\newtheorem{theorem}{Theorem}[section]
\newtheorem{corollary}{Corollary}
\newtheorem*{main}{Main Theorem}
\newtheorem{lemma}[theorem]{Lemma}
\newtheorem{proposition}{Proposition}
\newtheorem{conjecture}{Conjecture}
\newtheorem*{problem}{Problem}
\theoremstyle{definition}
\newtheorem{definition}[theorem]{Definition}
\newtheorem{remark}{Remark}
\newtheorem*{notation}{Notation}
\newcommand{\ep}{\varepsilon}
\newcommand{\eps}[1]{{#1}_{\varepsilon}}


%% Place the running title of the paper with 40 letters or less in []
 %% and the full title of the paper in { }.
\title[Running head with forty characters or less]
      {Instructions for preparing  \LaTeX\ File\\ for AIMS Journals }

 \email{editorial@aimsciences.org}
\begin{document}
\maketitle



Here below are important steps and instructions on how to prepare
your final \TeX\ file. Please read carefuly and follow them as close
as possible since it will expedite the publication process of your
paper. Thank you very much for your
cooperation.\\


\noindent\textbf{I Some First Steps}
\begin{enumerate}

\item \textbf{Step 1:}

\begin{compactenum}[i)]
\item For file prepared in \AmS-\LaTeX\ format:
Download the class file ``AIMS.cls" from
the following link and place it to the local folder  where your tex file resides.\\
\url{http://aimsciences.org/journals/tex-sample/AIMS.cls}.

\item For file prepared in \AmS-\TeX\ format: Please skip this step.

\end{compactenum}


\item \textbf{Step 2:}

\begin{compactenum}[i)]
\item For file prepared in \AmS-\LaTeX\ format:
Download the \textbf{\AmS-\LaTeX\ template}
 by clicking the following link and use it as
a template to prepare your tex file.\\
\url{http://aimsciences.org/journals/tex-sample/Template_1.tex}.



\item For file prepared in \AmS-\TeX\ format:
 Download the \textbf{\AmS-\TeX\ template}
 by clicking the following link and use it as
a template to prepare your tex file.\\
\url{http://aimsciences.org/journals/tex-sample/Template_2.tex}.

!!! \ \textbf{Important note:} Please read carefully all information
in these templates including those preceded by \% sign.

!!! \ \textbf{Important remark:} For special issue the info. of
``Communicated by the associate editor name" before abstract is not
needed.

\end{compactenum}


\item \textbf{Step 3:}

\begin{compactenum}[i)]
\item For file prepared in \AmS-\LaTeX\ format:
 Compile your paper using \AmS-\LaTeX\ command together with the class file
``AIMS.cls," making sure the compiling is error-free.

\item For file prepared in \AmS-\TeX\ format:
 Compile your paper using \AmS-\TeX\ command, making sure the compiling is error-free.

\end{compactenum}


\item \textbf{Step 4:} Print out your paper and check.
\begin{compactenum}[i)]
         \item Improve the quality of the figures until they are
          in high resolution and clear if the originals are  not.
         \item Make sure your paper includes the following important
         information\\
             1) Abstract\\
             2) Full address of each author including country name,
              placed beneath the title\\
             3) Email of each author, placed at the end of the paper\\
             4) Mathematics Subject Classification (MSC) number\\
             5) Key words and phrases\\
             6) End proof sign (a blank square aligned to right) for each proof
         \item Make sure the format of the references match those shown
         in the ``REFERENCES" section bellow, placing each Mathematical Reviews (MR) number right after  \verb+\+bibitem, e.g. \verb+\+bibitem\{A22\} (MR2082924).
 \end{compactenum}

\end{enumerate}

\vspace*{15pt} \noindent \textbf{II Instructions}

\begin{enumerate}


\item To produce a theorem, lemma, proposition, corollary,  conjecture,
etc., you need to use the standard command \verb+\+begin\{\dots\} to
start with, and \verb+\+end\{\dots\} to finish.  All texts in such
environment will be automatically \textbf{\emph{slanted}}.

\item For a definition, remark, or notation, please use the
standard commands \verb+\+begin\{\dots\} and \verb+\+end\{\dots\}.
However all texts in such a environment will be automatically
\textbf{upright}.

\item For all \textbf{proof}s, please use \verb+\+begin\{proof\} and
\verb+\+end\{proof\} commands. Do not define your own macros.

\item Make sure all \textbf{math formular numbers} are continuous.

\item Make sure all \textbf{lines}, \textbf{math formulas}
and \textbf{figures} are \textbf{\textbf{within the limit of 5
inches in width}}. In particular, formulas can not run to the right
of equation numbers. Never run out of the bound.

\item Important remarks on Figures.

Figures in PDF,  JPEG or PNG  format are preferred and should be
called for from the \TeX \, file and sent as separate files.

PostScript figures in EPS (Encapsulated PostScript) format are also
acceptable and should be called for from the \TeX \, file and sent
as separate files if one has too much difficulty with the EPS to PDF
format conversions. The preferred macro package for including EPS
figures files is the LaTeX graphicx package.

More importantly your paper will be compiled with PDFLaTex or
PDFTeXify and therefore please make every effort to ensure your
submissions are compatible, e.g. avoid calling \verb+\+psfrag in the
tex file.


%\begin{enumerate}
\begin{compactenum}[i)]
\item Notice that all color figures will be printed in black
and white in the AIMS journals. \textbf{Make sure that a black-white
printout of your figure is clear, with high resolution}.

\item All figures should be placed in the body of your paper and
before the REFERENCES section.

\item In a page with figures, there should be no unnecessary spare space.
Be sure that each page is fully occupied by figures and texts.

\item Make sure both the memory size and geometric size
of your figures is as small as possible, while they are \textbf{
clearly visible of all details.}  For example, a figure file with
size bigger than 1MB and a paper file with size bigger than 5MB
could cause some technical inconveniences. Papers with figures of
poor \textbf{resolution} cannot be accepted.

\item Figures should be  scalable.

\item Again, make sure that all \textbf{figures} are \textbf{\textbf{within the limit of 5
inches in width}}.  Never run out of the bound.
 \end{compactenum}
%\end{enumerate}

 See an example below.

\begin{figure}%[htp]
\begin{center}
  % Requires \usepackage{graphicx}
  % replace aims_logo.eps by your figure file name
  \includegraphics[width=1.5in]{aims_logo.eps}\\
  \caption{Here is the Caption of your figure}\label{AIMS}
  \end{center}
\end{figure}

\newpage

\item Important remarks on References.
      \begin{compactenum}[i)]

       \item  Please \textbf{DON'T list any references you never cited} in your paper.


      \item Make sure all reference \textbf{indexes} cited in the main text do exist in the REFERENCE section.

       \item  Please \textbf{DON'T use your own definition} in the REFERENCE section.


      \item Please  write each reference in three seperate
      \verb+\+newblocks. Place author
      names in first \verb+\+newblock, title in second \verb+\+newblock
      and the rest info. in the third \verb+\+newblock. You may leave first \verb+\+newblock blank if there is no
      author names, likewise leave second \verb+\+newblock blank if there is no
      title. Need examples? Please refer to the  references
      at the end of \textbf{AIMS template}.


%\verb+\+bibitem\{A11\}\\
%     \verb+\+newblock Author names,\\
%     \verb+\+newblock \emph{Title of the paper},\\
%     \verb+\+newblock Name of the Journal, \textbf{Vol.} (Year),
%     StaringPage--EndingPage.\\



      \item List papers in alphabetic order according to first
      authors.

      \item Always place the full first name first,
      then the middle name initial (optional), followed by the last name.
       If there are multiple authors, use the word `and' to connect the last two authors.
      See references  \cite {A11}, \cite{A22} and \cite{BFL} for details.

      \item If a reference is a paper in a journal,
      the title of the paper should be {\bf \emph{slanted}},
      which can be achieved by putting
      the title inside the braces: \verb+\+emph\{\dots\}.
      Only the first character in a paper's title is in capital.
      When you list a paper from a journal,
       please ignore the issue number since the page numbers and
      volume number yield sufficient information to identify the paper.
      Pay attention to the correct way of placing
       \textbf{volume number} (in bold face), \textbf{year},
      \textbf{starting page}--\textbf{ending page}. Note that
      starting and end page numbers should be separated by
      \textbf{two hyphen - - }
      See references  \cite{A22} and \cite{BFL} for details.

       \item If a reference is a book, the title should be put upright,
        the first letter of each word in the title should be capitalized.
        See reference
       \cite{rB} for details.

       \item If a reference is a paper in a conference proceeding,
        please see the sample reference \cite{Serrin}.

       \item  If the reference is a thesis, please see the sample reference \cite{Zeng}.

       \item
          It is very valuable to the readers if the references have direct
       links to the Mathematical Reviews (MR) Database and arXiv database.
        Therefore, it is necessary to add the MR number (published paper)
         or arXiv number (preprint paper) for each reference, whenever available.
      Your help in gathering such information (MR numbers or archive numbers)
       will ensure such information's accuracy and expedite the process
       of your paper's publication.
              The quickest way to find MR numbers is to search
              at the link:
                 \url{http://www.ams.org/mrlookup}
                  or \url{http://www.ams.org/mathscinet/}.
                 To find the archive number, please click the following
                 link:\break \url{http://www.arxiv.org}.
                 You may use any two-combination pertaining to a reference
                 to find the MR number of the paper, e.g.
                 last name of one of the authors and the title
                 (it is not necessary to put the full title);
                 or last name of one of the authors and the starting page number.
                 Please check for \textbf{accuracies} of your citation
                 against what is in the MR database and make your citation
                 \textbf{consistent} with the MR database. Concerning where to
                 put these MR or archive numbers, please refer to the references
               \cite{A22}, \cite{BFL}, \cite{rB}, \cite{Serrin} \cite{quas}
               and \cite{Te2008}
               at the end of \textbf{AIMS template}.

%    \end{enumerate}
       \end{compactenum}


\end{enumerate}

You can find various examples in \textbf{AIMS template}. Check it
out by clicking the
following link:\\
\indent\url{http://aimsciences.org/journals/tex-sample/Template_1.tex}

\begin{thebibliography}{99}

\bibitem{A11}
     \newblock FirstName (or FirstNameInitial.)  MiddleInitial. LastName, % first name middle initial. and then last name.  Only the first character in the paper title is capitalized.
     \newblock \emph{Title of the paper},
     \newblock Name of the Journal, \textbf{Volume} (Year), StaringPage--EndingPage.

% Example of paper with MR number:
\bibitem{A22} (MR2082924)
     \newblock C.  Wolf,
     \newblock \emph{A mathematical model for the propagation of a hantavirus in structured populations},
     \newblock Discrete Continuous Dynam. Systems - B, \textbf{4} (2004), 1065--1089.

% Example of multiple authors:
\bibitem{BFL} (MR1124979)
    \newblock Y. Benoist, P. Foulon and F. Labourie, % Use `and' connect the last two authors
    \newblock \emph{Flots d'Anosov a distributions stable et instable
     differentiables},
    \newblock (French) [Anosov flows with stable and unstable differentiable
     distributions], J. Amer. Math. Soc., \textbf{5} (1992), 33--74.

% Example of a book in the reference:
\bibitem{rB} (MR1301779)
     \newblock J.  Smoller,
     \newblock ``Shock Waves and Reaction-Diffusion Equations,"
     \newblock 2$^{nd}$ edition,  Springer-Verlag, New York, 1994.

% Example of an article in Academic Press or a book:
\bibitem{Serrin} (MR0402274)
    \newblock J. Serrin,
    \newblock  \emph{Gradient estimates for solutions of nonlinear elliptic
                     and parabolic equations},
    \newblock  in ``Contributions to Nonlinear Functional Analysis" (eds. E.H. Zarantonello and Author 2),
                Academic Press, (1971), 33--75.
% Example of a thesis:
\bibitem{Zeng}
    \newblock FirstName LastName,
    \newblock  ``Torsion Cycles and Set Theoretic Complete Intersection,"
    \newblock  Ph.D thesis, Washington University in St. Louis, 2006.

% Example of a preprint article with 7 digits archive number:
\bibitem{quas}
\newblock M. Entov, L. Polterovich and F. Zapolsky,
\newblock \emph{Quasi-morphisms
\newblock and the Poisson bracket}, preprint, \arXiv{math/0605406}.

% Example of a preprint article with 8 digits archive number:
\bibitem{Te2008}
\newblock A. Teplinsky,
\newblock \emph{Herman's theory revisited}, preprint,
\newblock \arXiv{0707.0078}.

% No author names:
\bibitem{Whho}
\newblock
\newblock ``SARS Expert Committee, SARS in Hong Kong: From Experience to
Action," Report of Hong Kong SARS Expert Committee,
\newblock 2003. Available from: \url{http://www.sars-expertcom.gov.hk/english/reports/reports.html}.

% No title:
\bibitem{tachet}
\newblock F. Abergel and R. Tachet,
\newblock
\newblock work in progress.

\end{thebibliography}

\medskip
% The data information below will be filled by AIMS editorial staff
Received xxxx 20xx; revised xxxx 20xx.
\medskip

\end{document}

\end{document}
